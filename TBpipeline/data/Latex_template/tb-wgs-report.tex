\documentclass[12pt]{article}
%%%%%%%%%%%%%%%%%%%%%%%%%%%%%%%%%%%%%%
%    Notes
%%%%%%%%%%%%%%%%%%%%%%%%%%%%%%%%%%%%%%
% This template was modified from the PeerJ article "Evidence-Based Design and Evaluation of a Whole Genome Sequencing Clinical Report for the Reference Microbiology Laboratory" by Gardy et al. It was developed for FHI by Ola Brynildsrud


% Version 1.0 : OCT. 2017

%%%%%%%%%%%%%%%%%%%%%%%%%%%%%%%%%%%%%%
%    Version Change Notes
%%%%%%%%%%%%%%%%%%%%%%%%%%%%%%%%%%%%%%

%-----------------------
%  Version 1.0
%
% Initial version

%-----------------------


%%%%%%%%%%%%%%%%%%%%%%%%%%%%%%%%%%%%%%
%     3rd party packages
%%%%%%%%%%%%%%%%%%%%%%%%%%%%%%%%%%%%%%
%\usepackage[utf8]{inputenc}
\usepackage[letterpaper, margin=0.5in,headheight=77pt,top=3.5cm]{geometry}
\usepackage{amsmath}
\usepackage{amssymb}
\usepackage{hyperref}
%\usepackage[table]{./packages/tex/latex/xcolor/xcolor}
\usepackage[table]{xcolor}
%\usepackage{./packages/tex/latex/tabu/tabu}
\usepackage{tabu}
\usepackage{fancyhdr}
%\usepackage{./packages/tex/latex/lastpage/lastpage}
\usepackage{lastpage}
%\usepackage{pageslts}
\usepackage{graphicx}
%\usepackage{./packages/tex/latex/collcell/collcell}
\usepackage{collcell}
%\usepackage{./packages/tex/latex/multirow/multirow}
\usepackage{multirow}
%\usepackage[skins]{./packages/tex/latex/tcolorbox/tcolorbox}
\usepackage[skins]{tcolorbox}
%\usepackage{fontspec}
%    \setmainfont{Lato}

\hypersetup{
  colorlinks   = true, %Colours links instead of using boxes to indicate links
  urlcolor     = black, %Colour for external hyperlinks
  linkcolor    = black %Colour for internal links
  citecolor   = red %Colour for citations
}
\newcolumntype{U}{>{\collectcell\MakeUppercase}l<{\endcollectcell}}

\tcbuselibrary{listings,breakable}

\newtcolorbox{reportSection}[2][]{%
  colback      = white!5!white,
  colframe     = white!75!black,
  fonttitle    = \bfseries,
  colbacktitle = white,
  title        = \textbf{\large{#2}},
  coltitle	   = black,
  arc = 0mm,
  opacityframe = 0.5,
  boxrule=1pt,
  attach boxed title to top left
           = {xshift=10pt, yshift=-8pt},
  boxed title style={%
    sharp corners, 
    rounded corners=northwest, 
    colback=white, 
    boxrule=0pt,
    titlerule=0mm},
  enhanced,
}

%--------------------------------------------
% Creating Header Style
%-------------------------------------------


%\renewcommand{\headrule}{\hbox to\headwidth{%
%    \color{gray}\leaders\hrule height \headrulewidth\hfill}}

%--------------------------------------------
% Creating Header with Text, Logo and Page Number
%-------------------------------------------

%-------------- HEADER CONTENT ------------

\pagestyle{fancy}
\fancyhf{}
\lhead{\large{\textbf{MYCOBACTERIUM TUBERCULOSIS \\ RAPPORT HELGENOMSSEKVENSERING}} \\ %ISO15189 reporting requirements stipulate that you provide "a clear, unambiguous identification of the examination including, where appropriate, the examination procedure"
\normalsize{IKKE AKKREDITERT}} %If you are using this as part of an accredited clinical pipeline, you may remove this text
\chead{}
\rhead{\includegraphics[scale=0.4]{imageFiles/fhi-logo-for-web-norsk-png}}
\renewcommand{\headrulewidth}{0pt} % no line in header area
%To accommodate new oversized header
%\setlength{\headheight}{4.5\baselineskip} %this is going to produce an error, but works - HOWEVER, it also makes the footer disappear

%-------------- FOOTER CONTENT ------------
% !!!! WARNING !!!!  OVERLEAF DOESN'T DO A GREAT JOB RUNNING THE \pageref COMMAND THAT WOULD LET US AUTOMATICALLY GET THE LAST PAGE #. SO, THE LAST PAGE IS HARD CODED HERE, THIS WILL CAUSE A PROBLEM IF YOUR REPORT IS LONGER THAN TWO PAGES!!!

\lfoot{Side \thepage\ av 2}
\input{include/footer.tex}%\rfoot{Isolat ID: 12345678910 | Dato: 2017-01-01 }

\begin{document}
%---------------------------------------------------------
%  TABLE WITH RELEVANT PATIENT INFORMATION
%--------------------------------------------------------
% Delete this line when including this report in your own pipeline vvvvv
%\noindent \footnotesize{\textbf{See %\url{https://github.com/amcrisan/TB-WGS-MicroReport} for how to automatically fill the contents of this template}}
%\vspace{3mm}
% ^^^^^^^^^^^^^^


\tabulinesep=5pt
\noindent 
\taburulecolor{lightgray}
\noindent \begin{tabu} to \textwidth {|XU||XU|}
\hline
\input{include/info.tex}
\end{tabu}
\vspace{5mm}

%---------------------------------------------------------
%  SECTION 1 : SUMMARY OF REPORT CONTENT
%---------------------------------------------------------

\begin{reportSection}{Oppsummering}
\vspace{1mm}

\input{include/oppsummering.tex}

\end{reportSection}
\vspace{5mm}

%---------------------------------------------------------
%  SECTION 2 : SUMMARY OF SPECIATION RESULTS
%---------------------------------------------------------

\begin{reportSection}{Typing} % Our design study indicated this was the preferred headline word for this section, in which speciation is summarised.
\vspace{1mm}
\input{include/typing.tex}
\end{reportSection}
\vspace{5mm}

%---------------------------------------------------------
%  SECTION 3 : SUMMARY OF DRUG SUSCEPTIBILITY RESULTS
%---------------------------------------------------------


\begin{reportSection}{Sensitivitet for antibiotika}
\vspace{1mm}
\tabulinesep=2pt
\noindent 
\taburulecolor{lightgray}
\noindent \begin{tabu} to \textwidth {X[1.1,l]X}
\multirow{5}{*}{\parbox{7.5cm}{\footnotesize{Resistens rapporteres n\aa r en kjent resistensmutasjon kan fastsl\aa es med h\o y konfidens. \textbf{``Ingen mutasjon detektert'' er ikke ensbetydende med sensitivitet ovenfor antibiotika}.}}}
\input{include/resistensbokser.tex}
\end{tabu}
\vspace{3mm} %% Add a Bit of space between the entities

%--------- Summary of drug susceptibility - table
\tabulinesep=5pt
\noindent 
\taburulecolor{lightgray}
\noindent \begin{tabu} to \textwidth {X[0.5,l]X[0.75,l]X[0.75,l]X[2,l]}
\hline
\input{include/resistens.tex}
\end{tabu}
\end{reportSection}

%---------------------------------------------------------
%  SECTION 4 : CLUSTER DETECTION RESULTS
%---------------------------------------------------------
\pagebreak
\begin{reportSection}{Deteksjon av smitteclustre}
\vspace{3mm}
\input{include/beslektede_oppsummering}
\vspace{2mm}

\tabulinesep=5pt
\noindent 
\taburulecolor{lightgray}
\noindent \begin{tabu} to \textwidth {XX}
% Set your own thresholds and choice of language here. You might also wish to populate this section with a pre-written narrative about the cluster your isolate belongs to, if available
\input{include/beslektede.tex}

\end{tabu}

\includegraphics[scale=0.65]{./imageFiles/tree}%{./imageFiles/Test}%

\end{reportSection}
\vspace{1mm}

%---------------------------------------------------------
%  SECTION 5 : ASSAY DETAILS 
%---------------------------------------------------------
\begin{reportSection}{Detaljer}
\vspace{1mm}
\tabulinesep=5pt
\noindent 
\taburulecolor{lightgray}
\noindent \begin{tabu} to \textwidth {|XU||XU|}
\hline
\input{include/pipelinedetaljer}

\end{tabu}
\vspace{1mm}

\end{reportSection}
\vspace{1mm}

%---------------------------------------------------------
%  SECTION 6 : ADDITIONAL COMMENTS
%---------------------------------------------------------
%Comments are for anything funny that was found by the 
\begin{reportSection}{Kommentarer}
% This is where any additional comments might go
\vspace*{20mm}

%\footnotesize{No additional comments for this report}
%\vspace{2mm}

%\textcolor{gray}{\footnotesize{\textit{\textbf{Standard Disclaimer:} Low frequency hetero-resistance below the limit of detection by sequencing may affect typing results.  The interpretation provided is based on the current understanding of genotype-phenotype relationships.}}}

\end{reportSection}
\vspace{1mm}


%---------------------------------------------------------
%  SECTION 7 : DISCLAIMER
%---------------------------------------------------------
% Note: Disclaimer is a general note that will appear on all reports. In contrast to Comments, which is for the the result of the individual patient. 
%\begin{reportSection}{Disclaimer}
% This is where any additional comments might go
%\vspace{1mm}

%\footnotesize{Low frequency hetero-resistance below the limit of detection by sequencing may affect typing results.  The interpretation provided is based on the current understanding of genotype-phenotype relationships.}
%\end{reportSection}
%\vspace{1mm}

%---------------------------------------------------------
%  SECTION 8 : AUTHORIZATION
%---------------------------------------------------------
\begin{reportSection}{Godkjent av}
% This section is required by ISO15189 standards
\vspace{1mm}
\tabulinesep=5pt
\noindent 
\taburulecolor{lightgray}
\noindent \begin{tabu} to \textwidth {|XU||XU|}
\hline
%--------------- v ADD YOUR TABLE CONTENTS BELOW v ---------------------
Signatur  &   & Navn  &   \\ \hline
Stilling   &   & Dato    &        \\ \hline
%--------------- ^ ADD YOUR TABLE CONTENTS ABOVE ^ ---------------------
\end{tabu}

\end{reportSection}



\end{document}
